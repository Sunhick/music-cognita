% This is "sig-alternate.tex" V2.1 April 2013
% This file should be compiled with V2.5 of "sig-alternate.cls" May 2012
%
% This example file demonstrates the use of the 'sig-alternate.cls'
% V2.5 LaTeX2e document class file. It is for those submitting
% articles to ACM Conference Proceedings WHO DO NOT WISH TO
% STRICTLY ADHERE TO THE SIGS (PUBS-BOARD-ENDORSED) STYLE.
% The 'sig-alternate.cls' file will produce a similar-looking,
% albeit, 'tighter' paper resulting in, invariably, fewer pages.
%
% ----------------------------------------------------------------------------------------------------------------
% This .tex file (and associated .cls V2.5) produces:
%       1) The Permission Statement
%       2) The Conference (location) Info information
%       3) The Copyright Line with ACM data
%       4) NO page numbers
%
% as against the acm_proc_article-sp.cls file which
% DOES NOT produce 1) thru' 3) above.
%
% Using 'sig-alternate.cls' you have control, however, from within
% the source .tex file, over both the CopyrightYear
% (defaulted to 200X) and the ACM Copyright Data
% (defaulted to X-XXXXX-XX-X/XX/XX).
% e.g.
% \CopyrightYear{2007} will cause 2007 to appear in the copyright line.
% \crdata{0-12345-67-8/90/12} will cause 0-12345-67-8/90/12 to appear in the copyright line.
%
% ---------------------------------------------------------------------------------------------------------------
% This .tex source is an example which *does* use
% the .bib file (from which the .bbl file % is produced).
% REMEMBER HOWEVER: After having produced the .bbl file,
% and prior to final submission, you *NEED* to 'insert'
% your .bbl file into your source .tex file so as to provide
% ONE 'self-contained' source file.
%
% ================= IF YOU HAVE QUESTIONS =======================
% Questions regarding the SIGS styles, SIGS policies and
% procedures, Conferences etc. should be sent to
% Adrienne Griscti (griscti@acm.org)
%
% Technical questions _only_ to
% Gerald Murray (murray@hq.acm.org)
% ===============================================================
%
% For tracking purposes - this is V2.0 - May 2012

\documentclass{sig-alternate-05-2015}


\begin{document}

% Copyright
\setcopyright{acmcopyright}
%\setcopyright{acmlicensed}
%\setcopyright{rightsretained}
%\setcopyright{usgov}
%\setcopyright{usgovmixed}
%\setcopyright{cagov}
%\setcopyright{cagovmixed}


% DOI
\doi{}

% ISBN
\isbn{}

%Conference
\conferenceinfo{}{Feb 25, 2015}

%
% --- Author Metadata here ---
\conferenceinfo{}{}
%\CopyrightYear{2007} % Allows default copyright year (20XX) to be over-ridden - IF NEED BE.
%\crdata{0-12345-67-8/90/01}  % Allows default copyright data (0-89791-88-6/97/05) to be over-ridden - IF NEED BE.
% --- End of Author Metadata ---

\title{Music Recommender and Genre classification system}
\subtitle{}
%
% You need the command \numberofauthors to handle the 'placement
% and alignment' of the authors beneath the title.
%
% For aesthetic reasons, we recommend 'three authors at a time'
% i.e. three 'name/affiliation blocks' be placed beneath the title.
%
% NOTE: You are NOT restricted in how many 'rows' of
% "name/affiliations" may appear. We just ask that you restrict
% the number of 'columns' to three.
%
% Because of the available 'opening page real-estate'
% we ask you to refrain from putting more than six authors
% (two rows with three columns) beneath the article title.
% More than six makes the first-page appear very cluttered indeed.
%
% Use the \alignauthor commands to handle the names
% and affiliations for an 'aesthetic maximum' of six authors.
% Add names, affiliations, addresses for
% the seventh etc. author(s) as the argument for the
% \additionalauthors command.
% These 'additional authors' will be output/set for you
% without further effort on your part as the last section in
% the body of your article BEFORE References or any Appendices.

\numberofauthors{3} %  in this sample file, there are a *total*
% of EIGHT authors. SIX appear on the 'first-page' (for formatting
% reasons) and the remaining two appear in the \additionalauthors section.
%
\author{
% You can go ahead and credit any number of authors here,
% e.g. one 'row of three' or two rows (consisting of one row of three
% and a second row of one, two or three).
%
% The command \alignauthor (no curly braces needed) should
% precede each author name, affiliation/snail-mail address and
% e-mail address. Additionally, tag each line of
% affiliation/address with \affaddr, and tag the
% e-mail address with \email.
%
% 1st. author
\alignauthor
Sunil BN\titlenote{Graduate Student}\\
       \affaddr{University of Colorado Boulder}\\
       \affaddr{Boulder, Colorado}\\
       \affaddr{United states}\\
       \email{suba5417@colorado.edu}
% 2nd. author
\alignauthor
Praveen Kumar Devaraj\titlenote{Graduate Student}\\
       \affaddr{University of Colorado Boulder}\\
       \affaddr{Boulder, Colorado}\\
       \affaddr{United states}\\
       \email{prde1873@colorado.edu}
% 3rd. author
\alignauthor
Suresh Kumar\titlenote{Graduate Student}\\
       \affaddr{University of Colorado Boulder}\\
       \affaddr{Boulder, Colorado}\\
       \affaddr{United states}\\
       \email{suku3476@colorado.edu}
}
% There's nothing stopping you putting the seventh, eighth, etc.
% author on the opening page (as the 'third row') but we ask,
% for aesthetic reasons that you place these 'additional authors'
% in the \additional authors block, viz.

% Just remember to make sure that the TOTAL number of authors
% is the number that will appear on the first page PLUS the
% number that will appear in the \additionalauthors section.

\maketitle
\begin{abstract}
In the recent years, automatic music recommendation systems are gaining popularity because a lot of music is now sold digitally. Most of the music recommender system are based on collaborative filtering or content based filtering. This project presents a hybrid music recommendation method that solves two of inherent problems in conventional methods(collaborative/content based filtering). Collaborative approach cannot recommend music that has no ratings and suffers from cold start. Where as content based filtering which recommends music based on the similarity of contents doesn't truly reveal the user preferences. Our approach combines both methods i.e based on user rating and content using discriminative model. We propose to train the collaborative filtering model such as Mlib(Machine learning library) and predict songs user might listen to. Find songs that are most similar to users' preferences and recommend them later.\\

We compare a conventional approaches of collaborative and content based filtering with our hybrid model, and evaluate the predictions quantitatively and qualitatively on the Million Song Data-set(MSD).
\end{abstract}


% We no longer use \terms command
%\terms{Theory}

\keywords{Hybrid model; collaborative filtering; content based filtering; Machine learning library(Mlib); Million song dataset, MSD}

\section{Motivation}
Given a huge collection of digital media like audio, video etc a recommneder system help users find and select items of their tastes. The Music recommneder system will present to the listeners a subset of the songs that are well suited to their interest based on mood/genre/time of day etc. Thus building a robust music recommendation and Genre classification system is complicated by the variety of songs, lyrics and very large collection of dataset etc. Further more most of the recommendation system work on either collaborative or content based filtering, both have their pros and cons. In this project we will consider best features of both the conventional methods to build a better and more accurate music recommender system.


\section{Literature Survey}
With the explosion of Internet in recent decades, Internet has become a major source for information such as audio, video, books, geography etc. The music also has drifted more and more towards the digital distribution of musical pieces through on-line stores like iTunes, Spotify, SoundCloud, Shazam and Google play music. As a result music categorization and recommendation has become a persistent problem. Music categorization allow on-line music to be classified based on the genre, lyrics etc. so that the users' can browse through the similar contents. And recommender system enables users' to discover new music that matches their tastes based on previously listened music.\\

Though much of the studies and research has been done in recommender system, still the problem here is complicated by the variety of tons of songs, different genres, social and geographical factors that influence the users. The number of music that can be recommended to the user is very large.\\

The problem is to organize the humongous amount of music available on-line into categories like based on genre, artist, most popular etc.MIR  techniques  have  been  developed  to  solve problems  such  as  genre  classification,  artist  identification,  and  instrument recognition. Since 2005, an annual evaluation event called Music Information Retrieval Evaluation eXchange (MIREX1) is held to facilitate the development of MIR algorithms\cite{DeepContentbased}.\\

Additionally, music recommender system is to help the users find and discover the music according to their tastes. A good recommender system should be able to learn the listening styles/preferences of users and generate play-list accordingly. Meanwhile, the development of recommender systems provides a great opportunity for industry  to  aggregate  the  users  who  are  interested  in  music.  More  importantly, it raises challenges for us to better understand and model users preferences in music.\\

Some music discovery websites such as Last.fm, Allmusic, Pandora and Shazam have successfully used these two approaches into reality. In the meantime, these websites provide an unique platform to retrieve rich and useful information for user studies.\\

\section{Proposed work}
The project would use a hybrid approach combining both collaborative filtering and content-based approaches. Collaborative filtering is based on the idea of determining user's preference using the listeners' historical data. Collaborative method recommend a song to user based on how other users rated the recommended song. For example, if a user listens to song A,B,C and another user listens to song B,C,D, then song D would be recommended to first user and song A would be recommended to the second user.\\

This approach does not need to have any information about the song. The key issue with this approach is that it need to have large user data and thus suffer from cold start. Also, new and unpopular song cannot be recommended.\\

Content based approach recommends music based on the properties and attributes of the music. The various kinds of information associated with music that could be used are tags, artist and albums information, lyrics, reviews, and audio single itself.

\subsection{Dataset}
The dataset which we will use is called Million song dataset. It is a freely-available collection of audio features and meta-data for a million contemporary popular music tracks. We would first load the user data and run a collaborative filtering methods on it such as Probabilistic Latent Semantic Analysis. This method would predict the songs the user might like.\\

The next step would be to use the attributes of the songs predicted in content approach, train a machine learning algorithm such as logistic regression, Support vector machine, Naive Bayes etc and then give the final recommendations. More advanced techniques like deep learning, neural nets etc could be taken up as a future work.

\subsection{Sub-Task}
The various sub-tasks which we would be implementing as part of this project are:
\begin{itemize}
    \item Recommending popular songs to the user.
    \item Automatic genre classification of songs.
    \item Automatic categorization of  songs with mood tags such as ``sad``, ``happy``, ``joyful``, ``romantic``, etc.
    \item Integration with existing music systems(iTunes, Google music, etc).
    \item Evaluation of our approaches using metrics like root mean squred error(RMS) etc.
    \item Comparison of various techniques used in music recommender system.
    \item Recommend songs based on what kind of music a user likes to listen during different time periods in a day. This would eventually require all the timestamps a particular song was listened to. This could be scoped for future work since the data is not available for this.
\end{itemize}

\subsection{Evaluation}
To evaluate how well our system would perform, we can make use of different metrics like Mean Squared Error, Root Mean Squared Error, precision and recall or DCG. We will split the input data into training set and test set. \\

The test set should be only a small fraction of the complete data. The idea is to hide the test set from the system and see how well the system is able to reproduce it. We will use the training set as input data to the recommender system to build the classifier. The system would then produce the recommendations, which we will compare to what the user actually listened to and use the above mentioned metrics to compute a accuracy of our model.


\section{Milestones}
Below are the proposed miles for Hybrid Music recommender system.\\\\
March 12th, 2016 - Implementation of the hybrid method for classifying the songs based on genres. Recommending popular songs/playlist to the users.\\

March 20th, 2016 - Automatic categorization of  songs with mood tags such as ``sad``, ``happy``, ``joyful``, ``romantic``, etc.\\

April 3rd, 2016 - Implementation of evaluation metrics and comparison of the techniques proposed with pre-existing recommendation systems.\\

April 15th, 2016 - Project documentation and reporting.\\

Future work - Implement deep Learning for Music Recommendation systems. Recommend songs based on what kind of music a user likes to listen during different time periods in a day.\\

\section{Peer Evaluation}

The peer evaluation was done in collaboration with project team working on - `Mining stack overflow data`. Authors: Nikhil Mahendra, Sachin Muralidhara, Aadish, Saurabh sood.\\

The goal of peer review is to help improve classmate's work by pointing out strengths and weaknesses that may not be apparent. During this session we had to answer several questions and few of the interesting ones are listed below.\\

What data is used to train the model for genre Classification? \\

As we know in Million songs dataset, we do not have an explicit attribute genre for all the songs in the dataset. There is a small dataset with genre attributes (the GTZAN dataset). We intend to classify the songs in MSD based on genre using a cross-modal retrieval framework which combines features of lyrics and audio available in MSD. Our likelihood-based training, combining audio sequence features and text features, is done using techniques like Hidden markov model(HMM) and Bag of words model for lyrics of the songs available in MSD.\\

What is unique about our approach in Hybrid recommendation? \\

Popular companies have been using collaborative filtering or content based filtering to come up with the recommendations. We propose a hybrid music recommender system that theoretically integrates collaborative data (rating scores of users) and content-based data (acoustic features of audio signals) to meet the our requirements. To integrate these methods (Collaborative and Content based filtering), we first represent user preferences by using content-based data and then make recommendations in a collaborative way by computing the similarities of the music based user preferences.\\

For this we use a probabilistic generative model, called a three-way aspect model, proposed by Popescul et al. which explains the probabilistic generative mechanism for the observed data (rating scores and acoustic features) by introducing a set of latent variables. As part of the generative mechanism, the model directly represents user preferences (latent favorite genres) estimated statistically with a theoretical proof. This estimation makes the method after predicting unknown rating scores by using content what techniques do we use to generate the recommendations.\\

What are the evaluation metrics? \\

To evaluate the performance of the system, we plan to make use of different metrics like Mean Squared Error, Root Mean Squared Error, precision and recall or DCG. We split the input data into training set and test set. The test set should be only a small fraction of the complete data. The idea is to hide the test set from the system and see how well the system is able to reproduce it. We use the training set as input data to the recommender system.\\

The system would produce the recommendations, which we compare with the user history and use the above mentioned metrics to compute a value. Also we would be comparing the accuracy measure with previously existing systems whose accuracy is expressed in terms of RMS Error.

\section{Conclusions}

In this project, we propose a hybrid model for music recommendation, which combines features from audio and lyrics for the task of music genre classification and recommendation. HMM's, lyrics bag-of-words, and loudness/tempo features would be used. Here we proposed a hybrid music recommender system that ranks musical pieces by comprehensively considering collaborative and content-based data, i.e., rating scores derived from users and acoustic features derived from audio signals.\\

We proposed a probabilistic generative model called a three-way aspect model to create our model which can theoretically explain the generative mechanism for both kinds of the observed data.One possible interpretation of the generative mechanism is that a user stochastically selects a genre according to his or her preferences and then the genre stochastically generates a musical piece and an acoustic feature.\\

Thus the joint probability distribution over users, pieces, and features is decomposed into three independent distributions, which are respectively conditioned by genres.This allows us to incrementally train the aspect model according to the increase in users and rating scores at low computational cost.\\

Thus main gist of the project can be summarized below -\\

\begin{itemize}
    \item From features of audio and lyrics predict the genre of the music which is used for classification and recommendation.
    \item A Hybrid Recommender system which offers more accurate recommendations than existing systems.
    \item Apply Deep Learning techniques for computing the similarity in musicals (Future work).
    \item Compare the accuracy of the the different techniques with few evaluation metrics like RMS error etc.
\end{itemize}

\begin{thebibliography}{}
\bibitem{Genre classification} 
Dawen Liang, Haijie Gu and Brendan O\`Connor.\\
\textit{Music Genre Classification with the Million Song Dataset}. \\
 
\bibitem{Hybrid} 
Kazuyoshi Yoshii, Masataka Goto, Kazunori Komatani, Tetsuya Ogata and Hiroshi G. Okuno.
\textit{An Efficient Hybrid Music Recommender System Using an Incrementally Trainable Probabilistic Generative Model}. \\

\bibitem{DeepContentbased}
A$\bar{a}$ron van den Oord, Sander Dieleman, Benjamin Schrauwen. \\
Electronics and Information Systems department (ELIS), Ghent University
\textit{Deep content-based music recommendation}

\bibitem{SurveyMRS}
Yading Song, Simon Dixon, and Marcus Pearce.\\
Centre for Digital Music. Queen Mary University of London.\\
\textit{A Survey of Music Recommendation Systems and Future Perspectives}

\bibitem{SurveyMRS}
Asela Gunawardana and Christopher Meek.\\
Microsoft Research, One Microsoft Way, Redmond, WA 98052.\\
\textit{A Unified Approach to Building Hybrid Recommender Systems}

\end{thebibliography}

%\balancecolumns % GM June 2007
% That's all folks!
\end{document}
